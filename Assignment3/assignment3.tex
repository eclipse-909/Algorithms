%%%%%%%%%%%%%%%%%%%%%%%%%%%%%%%%%%%%%%%%%
%
% CMPT 435L 112
% Fall Semester
% Assignment 3
%
%%%%%%%%%%%%%%%%%%%%%%%%%%%%%%%%%%%%%%%%%

%%%%%%%%%%%%%%%%%%%%%%%%%%%%%%%%%%%%%%%%%
% Short Sectioned Assignment
% LaTeX Template
% Version 1.0 (5/5/12)
%
% This template has been downloaded from: http://www.LaTeXTemplates.com
% Original author: % Frits Wenneker (http://www.howtotex.com)
% License: CC BY-NC-SA 3.0 (http://creativecommons.org/licenses/by-nc-sa/3.0/)
% Modified by Alan G. Labouseur  - alan@labouseur.com
%
%%%%%%%%%%%%%%%%%%%%%%%%%%%%%%%%%%%%%%%%%

%----------------------------------------------------------------------------------------
%	PACKAGES AND OTHER DOCUMENT CONFIGURATIONS
%----------------------------------------------------------------------------------------

\documentclass[letterpaper, 10pt,DIV=13]{scrartcl} 

\usepackage[T1]{fontenc} % Use 8-bit encoding that has 256 glyphs
\usepackage[english]{babel} % English language/hyphenation
\usepackage{amsmath,amsfonts,amsthm,xfrac} % Math packages
\usepackage{sectsty} % Allows customizing section commands
\usepackage{graphicx}
\usepackage[lined,linesnumbered,commentsnumbered]{algorithm2e}
\usepackage{listings}
\usepackage{parskip}
\usepackage{lastpage}
\usepackage{minted}

\allsectionsfont{\normalfont\scshape} % Make all section titles in default font and small caps.

\usepackage{fancyhdr} % Custom headers and footers
\pagestyle{fancyplain} % Makes all pages in the document conform to the custom headers and footers

\fancyhead{} % No page header - if you want one, create it in the same way as the footers below
\fancyfoot[L]{} % Empty left footer
\fancyfoot[C]{} % Empty center footer
\fancyfoot[R]{page \thepage\ of \pageref{LastPage}} % Page numbering for right footer

\renewcommand{\headrulewidth}{0pt} % Remove header underlines
\renewcommand{\footrulewidth}{0pt} % Remove footer underlines
\setlength{\headheight}{13.6pt} % Customize the height of the header

\numberwithin{equation}{section} % Number equations within sections (i.e. 1.1, 1.2, 2.1, 2.2 instead of 1, 2, 3, 4)
\numberwithin{figure}{section} % Number figures within sections (i.e. 1.1, 1.2, 2.1, 2.2 instead of 1, 2, 3, 4)
\numberwithin{table}{section} % Number tables within sections (i.e. 1.1, 1.2, 2.1, 2.2 instead of 1, 2, 3, 4)

\setlength\parindent{0pt} % Removes all indentation from paragraphs.

\binoppenalty=3000
\relpenalty=3000

%----------------------------------------------------------------------------------------
%	TITLE SECTION
%----------------------------------------------------------------------------------------

\newcommand{\horrule}[1]{\rule{\linewidth}{#1}} % Create horizontal rule command with 1 argument of height

\title{	
   \normalfont \normalsize 
   \textsc{CMPT 435L 112 - Fall 2024 - Dr. Labouseur} \\[10pt] % Header stuff.
   \horrule{0.5pt} \\[0.25cm] 	% Top horizontal rule
   \huge Assignment Three  \\     	    % Assignment title
   \horrule{0.5pt} \\[0.25cm] 	% Bottom horizontal rule
}

\author{Ethan Morton \\ \normalsize Ethan.Morton1@Marist.edu}

\date{\normalsize\today} 	% Today's date.

\begin{document}
\maketitle % Print the title

%----------------------------------------------------------------------------------------
%   start PROBLEM ONE
%----------------------------------------------------------------------------------------
\section{Graphs}
Template specializations are used to implement three types of undirected, non-weighted graphs:
matrix, adjacency list, and linked objects. All graphs use some data structure to keep track of
the vertex IDs and the edges between them. All graphs also use a lookup-table which allows
you to look up the vertex in that data structure in constant time when given the vertex ID.
The lookup-table is defined as unordered\_map<string, int> id\_lookup; for each implementation.
The int represents the index used to find the vertex in a vector. This index auto-increments when
a new vertex ID is added. This means the order of vertices in the vector is the same as which they
appear in the graph file.

\subsection{Matrix}
The data structure for the matrix is vector<pair<string, vector<bool>>>. To break it down, the outer vector
represents each row in the matrix. The pair holds the vertex ID in the string, and the vector<bool> holds
whether there is an edge to the vertex with that index in the vector. Initially, I was going to use
vector<vector<uint8\_t>> and do bit manipulation myself to make the matrix memory efficient, but
C++ does this optimization by default when using vector<bool>.

Since this is a non-weighted graph, each edge only needs to store whether it exists, hence the
bool, as opposed to a numeric type. Additionally, since the graph is undirected, it only needs to
store half of the edges.

The edge between 1 and 2 is the same as the edge between 2 and 1, so we only need to track this
one time. To make this optimization, the number of columns in each row is equal to that row
number + 1. We have to include the +1 since a vertex can have an edge with itself. This gives us a
triangle-matrix instead of a square-matrix.

When doing this optimization, we also need to consider what happens when we add an edge and the
vertex IDs are given in an order we don't want. Since half the matrix is missing, we would be
indexing the vector outside of its range, which is undefined behavior (or maybe C++ throws an
exception... I can't remember). We can simply get around this by forcing the row to be greater
than or equal to the column, and swapping them if this is false.

\subsubsection{Add Vertex}

graph.hpp
\begin{minted}[linenos=true, firstnumber=45, fontsize=\large, breaklines]{C++}
void add_vertex(const string& id) {
    const int index = static_cast<int>(id_lookup.size());
    id_lookup[id] = index;
    pair<string, vector<bool>> p(id, vector<bool>(matrix.size() + 1));
    matrix.emplace_back(p);
}
\end{minted}

Adding a vertex is constant time. It auto-increments the index based on the size of the lookup table, then
inserts the ID-index pair into the table, then pushes the ID-row pair into the matrix.

\subsubsection{Add Edge}

graph.hpp
\begin{minted}[linenos=true, firstnumber=52, fontsize=\large, breaklines]{C++}
void add_edge(const string& id0, const string& id1) {
        uint8_t r = id_lookup[id0];
        uint8_t c = id_lookup[id1];
        //In an undirected graph, we only need to keep track of the edge in one spot.
        //This is done by forcing the ordering of the given vertex IDs to always be a specific way.
        //In this case, the row must not be less than the column.
        //This is why we can use the triangle shape shown in the Graph<Matrix> documentation.
        if (r < c) {
            std::swap(r, c);
        }
        matrix[r].second[c] = true;
}
\end{minted}

Adding an edge is constant time. It gets the index from the lookup-table, forces the ordering of the index,
then sets the boolean to true.

\subsubsection{Print}

graph.hpp
\begin{minted}[linenos=true, firstnumber=65, fontsize=\large, breaklines]{C++}
void print() const {
    printf(" ");
    //print X-axis vertex IDs
    for (const pair<string, vector<bool>>& pair : matrix) {
        if (pair.first.length() == 1) {
            printf("  %s", pair.first.c_str());
        } else {
            printf(" %s", pair.first.c_str());
        }
    }
    printf("\n");
    //Print each row
    for (const pair<string, vector<bool>>& pair : matrix) {
        //print vertex ID
        if (pair.first.length() == 1) {
            printf(" %s ", pair.first.c_str());
        } else {
            printf("%s ", pair.first.c_str());
        }
        //Print edges
        for (const bool col : pair.second) {
            if (col) {
                printf("1  ");
            } else {
                printf(".  ");
            }
        }
        printf("\n");
    }
}
\end{minted}

The time complexity for printing the matrix is $O(\frac{n(n+1)}{2})$ or $O(n^2)$. Rows are printed in the
order the appear in the graphs file. It should come out to look like a triangle.

\subsection{Adjacency List}
The adjacency list specialization is actually very similar to the matrix specialization. However, instead
of storing whether there is an edge between each pair of vertices, it only stores the index of the vertices
that each vertex has an edge with. The data structure for the adjacency list is
vector<pair<string, vector<int>>>. The difference between this and the matrix is the int instead of bool.

Like the matrix, the outer vector represents the rows. The string is the vertex ID, and the vector<int> is
a list of the index of each vertex that this one has an edge with.

Also like the matrix, the adjacency list uses a memory-optimization thanks to the graph being undirected.
We only need to store each edge once. When adding an edge, we can look at the indices given in the function
arguments. We can make sure the first one is greater than or equal to the second one, and swap them if this
is false. When we update the table, we are always indexing by the larger index, and inserting the smaller
index into that vertex's adjacency list. Vertex 2 will reference vertex 1 as a neighbor, but not the other
way around.

If we wanted to search whether vertex 1 is adjacent to vertex 2, this memory-optimization would slightly
slow down the search time, but it would still remain $O(n)$. Fortunately we aren't doing this search, so we
can do this optimization cost-free.

\subsubsection{Add Vertex}

graph.hpp
\begin{minted}[linenos=true, firstnumber=115, fontsize=\large, breaklines]{C++}
void add_vertex(const string& id) {
    id_lookup[id] = static_cast<int>(id_lookup.size());
    pair<string, vector<int>> p(id, vector<int>());
    adj_list.emplace_back(p);
}
\end{minted}

Similar to the matrix specialization, adding a vertex is constant time, and it just inserts the ID-index
pair into the table, and pushes the ID-list pair into the list.

\subsubsection{Add Edge}

graph.hpp
\begin{minted}[linenos=true, firstnumber=121, fontsize=\large, breaklines]{C++}
void add_edge(const string& id0, const string& id1) {
    int index0 = id_lookup[id0];
    int index1 = id_lookup[id1];
    //In an undirected graph, we only need to keep track of the edge in one spot.
    //This is done by forcing the ordering of the given vertex IDs to always be a specific way.
    //In this case, index0 must not be less than index1.
    //This way, the adjacency list is always indexed by the same number when given the same two vertices.
    if (index0 < index1) {
        std::swap(index0, index1);
    }
    adj_list[index0].second.emplace_back(index1);
}
\end{minted}

Adding an edge is done in constant time. It forces the ordering of the index if necessary, then uses
the larger index to index into the outer vector, then pushes the smaller index to the back of the inner
vector.

\subsubsection{Print}

graph.hpp
\begin{minted}[linenos=true, firstnumber=134, fontsize=\large, breaklines]{C++}
void print() const {
    for (const pair<string, vector<int>>& row : adj_list) {
        //print vertex ID
        printf("%s |", row.first.c_str());
        //print vertex ID of adjacent vertices
        for (const int col : row.second) {
            printf(" %s,", adj_list[col].first.c_str());
        }
        printf("\n");
    }
}
\end{minted}

\subsection{Linked Objects}
The linked objects specialization is a bit different from the other two. This one keeps a vector of
vertices. The Vertex struct has string ID and vector<const Vertex*> neighbors. The pointers in neighbors
are pointers to the other vertices in the graph's vector.

\subsubsection{Add Vertex}

graph.hpp
\begin{minted}[linenos=true, firstnumber=159, fontsize=\large, breaklines]{C++}
void add_vertex(const string& id) {
    id_lookup[id] = static_cast<int>(id_lookup.size());
    Vertex v = {id, vector<const Vertex*>()};
    vertices.emplace_back(v);
}
\end{minted}

Similar to the other specializations, adding a vertex is done in constant time. The lookup table is updated
and the vertex is added to the vertex vector.

\subsubsection{Add Edge}

graph.hpp
\begin{minted}[linenos=true, firstnumber=165, fontsize=\large, breaklines]{C++}
void add_edge(const string& id0, const string& id1) {
    Vertex* v0 = &vertices[id_lookup[id0]];
    Vertex* v1 = &vertices[id_lookup[id1]];
    v0->neighbors.emplace_back(v1);
    v1->neighbors.emplace_back(v0);
}
\end{minted}

Adding an edge is done in constant time. Unlike the other two specializations, this one requires both
vertices to be updated. The IDs are used to get the indices in the lookup table, and the addresses of those
vertices are pushed into the other vertex's neighbors vector.

\subsubsection{Breadth-First Traversal}

graph.hpp
\begin{minted}[linenos=true, firstnumber=172, fontsize=\large, breaklines]{C++}
void print_breadth_first() const {
    if (vertices.empty()) {return;}
    unordered_set<const Vertex*> processed;
    queue<const Vertex*> q;
    const Vertex* v0 = &vertices[0];
    processed.insert(v0);
    q.push(v0);
    while (!q.empty()) {
        const Vertex* curr = q.front();
        q.pop();
        printf("%s, ", curr->id.c_str());
        for (const Vertex* n : curr->neighbors) {
            if (!processed.contains(n)) {
                q.push(n);
                processed.insert(n);
            }
        }
    }
    printf("\n");
}
\end{minted}

Breadth-First traversal has a time complexity of $O(V + E)$ because it has to visit every vertex, and it
needs to check every edge to see if a vertex has been processed. The function uses an unordered\_set to
keep track of all the pointers that have been processed. The algorithm starts by printing the vertex at
index 0 in the vector. It uses a queue to keep track of which "layer" of vertices it's on. Once the first
vertex is printed, it runs through all of its neighbors and prints those. The neighbors are stored in the
queue, so in the next iteration, it goes through the items in the queue and prints it. A queue is used to
print out "layers" at a time, where a layer represents the neighbors of the previous layer.

\subsubsection{Depth-First Traversal}

graph.hpp
\begin{minted}[linenos=true, firstnumber=193, fontsize=\large, breaklines]{C++}
void print_depth_first() const {
    unordered_set<const Vertex*> processed;
    dft(&vertices[0], &processed);
    printf("\n");
}

private:
void dft(const Vertex* v, unordered_set<const Vertex*>* processed) const {
    if (!processed->contains(v)) {
        printf("%s, ", v->id.c_str());
        processed->insert(v);
    }
    for (const Vertex* n : v->neighbors) {
        if (!processed->contains(n)) {
            dft(n, processed);
        }
    }
}
\end{minted}

Depth-First traversal also has a time complexity of $O(V + E)$ for the same reason as breadth-first
traversal. Like BFT, DFT uses an unordered\_set to track the processed vertex pointers. BFT prints all the
neighbors of a vertex before moving on. DFT uses the call stack to move from vertex to vertex with each
call. It recursively prints the neighbor of the vertex, then its neighbor, then its neighbor, etc. It
processes the next neighbor of a vertex once the function returns that printed the previous neighbor.

\section{Binary Search Tree}
This is just a standard binary search tree with functions to insert, search, and perform a depth-first
traversal.

\subsection{Insertion}

binary\_search\_tree.cpp
\begin{minted}[linenos=true, firstnumber=24, fontsize=\large, breaklines]{C++}
void BinarySearchTree::insert(const string data) {
    printf("Inserting %s - Root", data.c_str());
    if (root == nullptr) {
        root = new Node(data);
        printf("\n");
        return;
    }
    Node* curr = root;
    while (curr != nullptr) {
        if (data < curr->data) {
            printf(",L");
            if (curr->left == nullptr) {
                curr->left = new Node(data);
                break;
            }
            curr = curr->left;
        } else {
            printf(",R");
            if (curr->right == nullptr) {
                curr->right = new Node(data);
                break;
            }
            curr = curr->right;
        }
    }
    printf("\n");
}
\end{minted}

Insertion is a $O(log_2(n))$ operation because it has to do a binary search through the tree to find the
location that the new node will be placed. It's actually slightly more because the tree likely won't be balanced. During the search, L or R is printed out to show the path of the search.

\subsection{Depth-First Traversal}

binary\_search\_tree.cpp
\begin{minted}[linenos=true, firstnumber=52, fontsize=\large, breaklines]{C++}
void traverse_helper(const Node* curr) {
    if (curr == nullptr) {return;}
    traverse_helper(curr->left);
    printf(" - %s", curr->data.c_str());
    traverse_helper(curr->right);
}

void BinarySearchTree::depth_first_traverse() const {
    printf("Depth-first traversal:");
    traverse_helper(root);
    printf("\n");
}
\end{minted}

Depth-first traversal uses recursion to print out each element in the BST in order. It starts by making
recursive calls to the left node, then when the function calls return, it starts printing out the nodes.
Then it makes recursive calls to the right. This is how it prints from left to right.

\subsection{Search}

binary\_search\_tree.cpp
\begin{minted}[linenos=true, firstnumber=65, fontsize=\large, breaklines]{C++}
int BinarySearchTree::search(const string target) const {
    printf("Searching %s - Root", target.c_str());
    int comparisons = 0;
    const Node* curr = root;
    while (curr != nullptr) {
        comparisons++;
        if (target == curr->data) {
            return comparisons;
        } else if (target < curr->data) {
            printf(",L");
            curr = curr->left;
        } else {
            printf(",R");
            curr = curr->right;
        }
    }
    return -1;
}
\end{minted}

This is a binary search function for the binary search tree. It has a time complexity of $O(log_2(n))$.
The average number of comparisons observed for searching for the magicitems-find-in-bst.txt items is
11.952381. $log_2(666)=9.379378$. The reason for this discrepancy is that the tree is unbalanced, and many
searches require more than 10 comparisons. The way it searches is by iterating down a specific branch until
it finds the target. If it hasn't found the target, it will move to the left if the target is less than the
current node, or it will move to the right if the target is greater than or equal to the current node.

\section{Putting it All Together}

\subsection{Verifying Graph Files}

main.cpp
\begin{minted}[linenos=true, firstnumber=30, fontsize=\large, breaklines]{C++}
int main(int argc, char* argv[]) {
    //Make a list of all the graph files to use
    vector<string> graph_files;
    if (argc == 1) {
        graph_files.emplace_back("./graphs1.txt");
    } else {
        for (int i = 1; i < argc; i++) {
            struct stat buffer;
            // File exists and is a regular file
            if (stat(argv[i], &buffer) == 0 && (buffer.st_mode & S_IFREG) != 0) {
                graph_files.emplace_back(argv[i]);
            } else {
                fprintf(stderr, "Error - invalid file path: %s\n", argv[i]);
                return 1;
            }
        }
    }

    //...
}
\end{minted}

To make it easier to run the program with different graph files, I have made it so you can list the files
in the arguments. See README.md for details.

\subsection{Graphs}

main.cpp
\begin{minted}[linenos=true, firstnumber=45, fontsize=\large, breaklines]{C++}
int main(int argc, char* argv[]) {
    //...

    //Do graph stuff for each graph file
    string line;
    for (const string& file : graph_files) {
        std::ifstream graph_file(file);
        if (!graph_file.is_open()) {
            fprintf(stderr, "Unable to open file %s\n", file.c_str());
            return 1;
        }
        Graph<Matrix> graph_m;
        Graph<AdjList> graph_al;
        Graph<vector<Vertex>> graph_lo;
        bool init = false;
        int lineNum = 0;
        while (std::getline(graph_file, line)) {
            lineNum++;
            //ignore comments
            if (size_t pos = line.find("--"); pos != std::string::npos) {
                line = line.substr(0, pos);
            }
            //skip empty lines
            if (line.empty()) {continue;}

            //parse tokens
            std::istringstream iss(line);
            std::string command;
            iss >> command;
            if (command == "new") {
                std::string graph;
                iss >> graph;
                if (graph == "graph") {
                    if (init) {
                        print_graphs(&graph_m, &graph_al, &graph_lo);
                    }
                    graph_m = Graph<Matrix>();
                    graph_al = Graph<AdjList>();
                    graph_lo = Graph<vector<Vertex>>();
                    init = true;
                } else {
                    fprintf(stderr, "Error - unrecognized token: %s\nLine number: %d\n", graph.c_str(), lineNum);
                    return 1;
                }
            } else if (command == "add") {
                std::string type;
                iss >> type;
                if (type == "vertex") {
                    if (string vertex; iss >> vertex) {
                        graph_m.add_vertex(vertex);
                        graph_al.add_vertex(vertex);
                        graph_lo.add_vertex(vertex);
                    } else {
                        fprintf(stderr, "Error - failed to parse vertex id as int\nLine Number: %d\n", lineNum);
                        return 1;
                    }
                } else if (type == "edge") {
                    char dash;
                    if (string vertex0, vertex1; iss >> vertex0 >> dash >> vertex1 && dash == '-') {
                        graph_m.add_edge(vertex0, vertex1);
                        graph_al.add_edge(vertex0, vertex1);
                        graph_lo.add_edge(vertex0, vertex1);
                    } else {
                        fprintf(stderr, "Error - failed to parse vertex id as int\nLine Number: %d\n", lineNum);
                        return 1;
                    }
                } else {
                    fprintf(stderr, "Error - unrecognized token: %s\nLine Number%d\n", type.c_str(), lineNum);
                    return 1;
                }
            } else {
                fprintf(stderr, "Error - unrecognized token: %s\nLine Number %d\n", command.c_str(), lineNum);
                return 1;
            }
        }
        print_graphs(&graph_m, &graph_al, &graph_lo);
        graph_file.close();
    }

    //...
}
\end{minted}

main.cpp
\begin{minted}[linenos=true, firstnumber=18, fontsize=\large, breaklines]{C++}
void print_graphs(const Graph<Matrix>* graph_m, const Graph<AdjList>* graph_al, const Graph<vector<Vertex>>* graph_lo) {
    printf("Graph with Matrix:\n");
    graph_m->print();
    printf("\nGraph with Adjacency List:\n");
    graph_al->print();
    printf("\nGraph with Linked Objects - Depth First:\n");
    graph_lo->print_depth_first();
    printf("\nGraph with Linked Objects - Breadth First:\n");
    graph_lo->print_breadth_first();
    printf("\n");
}
\end{minted}

This chunk of code is responsible for going through each graph file and doing the following:
\begin{itemize}
  \item Going line-by-line in the text file
  \item Parsing and tokenizing
  \item Handling the few possible tokens in a syntax tree
\end{itemize}

Here are the instructions handled in the syntax tree:
\begin{itemize}
  \item "new graph" or reaching the end of the file calls the print\_graphs function to print the 3 graphs.
  \item "add vertex \#" calls the add\_vertex function on the 3 graphs. \# represents the vertex ID.
  \item "add edge \# - \#" calls the add\_edge function on the 3 graphs. The \#'s represent the vertex IDs.
  \item Unrecognized tokens will cause the program to return early with an error.
\end{itemize}

\subsection{Binary Search Tree Insertion and Traversal}

main.cpp
\begin{minted}[linenos=true, firstnumber=121, fontsize=\large, breaklines]{C++}
int main(int argc, char* argv[]) {
    //...

    //Read lines from magicitems.txt and insert into BST
    std::ifstream magicitems("../magicitems.txt");
    if (!magicitems.is_open()) {
        fprintf(stderr, "Unable to open magicitems.txt\n");
        return 1;
    }
    BinarySearchTree bst;
    while (std::getline(magicitems, line)) {
        bst.insert(line);
    }
    magicitems.close();
    printf("\n");
    
    //print items in order
    bst.depth_first_traverse();
    printf("\n");

    //...
}
\end{minted}

Lines from magicitems.txt are read and inserted into the BST. Then depth\_first\_search is called on it.

\subsection{Binary Search Tree Searching}

main.cpp
\begin{minted}[linenos=true, firstnumber=138, fontsize=\large, breaklines]{C++}
int main(int argc, char* argv[]) {
    //...
    
    //Read lines from magicitems-find-in-bst.txt and search BST
    std::ifstream file("./magicitems-find-in-bst.txt");
    if (!file.is_open()) {
        fprintf(stderr, "Unable to open magicitems-find-in-bst.txt\n");
        return 1;
    }
    int comparisons = 0;
    int count = 0;
    while (std::getline(file, line)) {
        int c = bst.search(line);
        if (c == -1) {
            fprintf(stderr, ", Not found after %d comparisons\n", c);
            return 1;
        }
        printf(", Found after %d comparisons\n", c);
        comparisons += c;
        count++;
    }
    file.close();
    printf("Average comparisons: %f\n", static_cast<float>(comparisons) / static_cast<float>(count));

    return 0;
}
\end{minted}

Items are read from magicitems-find-in-bst.txt and passed into the BST's search function. Comparisons for
each search are returned and averaged. 

\end{document}